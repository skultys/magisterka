\chapter{Zastosowanie algorytmu quantum EA dla zagadnienia QAP}
Sposób w jaki działają algorytmy genetyczne, otwarty schemat działania, skłania do tworzenia wielu modyfikacji. Zmianom mogą podlegać operatory krzyżowani, mutacji, sposób kodowania rozwiązania. Można dodawać też nowe operatory o działaniu nieobjętym przez tradycyjne operatory. Z tego powodu, na przestrzeni lat, pojawia się wiele publikacji na temat algorytmów genetycznych i nowych sposób podejścia do tego tematu. W jednej z takich publikacji, autorzy Jinwei Gu, Xingsheng Gu i Manzhan Gu zaproponowali algorytm o nazwie \" a novel parallel quantum genetic algorithm\" i przedstawili jego wykorzystanie dla problemu szeregowania zadań. Jednakże algorytm ten nadaje się także dla innych zastosowań, do jakich należy na przykład problem QAP.

\subsection{Opis algorytmu}
Głównym elementem tego algorytmu, który wyróżnia go spośród innych, jest sposób kodowania rozwiązań. Z powodu  

\label{cha:qap_ea}