\chapter{Zastosowanie algorytmu quantum EA dla zagadnienia QAP}
\label{cha:qap_ea}
Sposób w jaki działają algorytmy ewolucyjne, ich otwarty schemat działania, skłania do tworzenia wielu modyfikacji. Przykładowo, w kontekście algorytmów genetycznych, zmianom mogą podlegać operatory krzyżowania, mutacji, sposób kodowania rozwiązania. Można dodawać też nowe operatory o działaniu nieobjętym przez tradycyjne operatory. Z tego powodu, na przestrzeni lat, pojawia się wiele publikacji na temat algorytmów ewolucyjnych i nowych sposób podejścia do tego tematu. W jednej z takich publikacji \cite{NPQGA}, autorzy Jinwei Gu, Xingsheng Gu i Manzhan Gu zaproponowali algorytm o nazwie \textit{,,a novel parallel quantum genetic algorithm'' - NPQGA} i przedstawili jego wykorzystanie dla problemu szeregowania zadań.  Algorytm ten należy do grupy tak zwanych kwantowych algorytmów ewolucyjnych i nadaje się także dla innych zastosowań do jakich należy na przykład problem QAP.

\section{Opis algorytmu}
Główną cechą kwantowych algorytmów ewolucyjnych jest zastosowanie w nich bitów kwantowych - kubitów. Wykorzystywane są one do reprezentacji rozwiązań algorytmów. Kubit w danym momencie może reprezentować teoretycznie nieskończenie wiele stanów będących superpozycją stanu $0$ i $1$. Obserwacja bitu kwantowego pozwala dopiero na jednoznaczną ocenę jego stanu. Stan kubitu może być reprezentowany przez równanie:
\newline
\begin{equation}
|\psi\rangle=\alpha|0\rangle+\beta|1\rangle
\end{equation}
\newline
gdzie $|\alpha|^2$ jest prawdopodobieństwem, że kubit znajduje się w stanie $0$, oraz $|\beta|^2$ jest prawdopodobieństwem, że kubit jest w stanie $1$. $\alpha$ i $\beta$ są liczbami zespolonymi. Obie liczby są znormalizowane, co znaczy, że:
\newline
\begin{equation}
|\alpha|^2+|\beta|^2=1
\end{equation}
\newline 

Kubit jest więc najmniejszą jednostką informacji w tych algorytmach i jest reprezentowany poprzez parę liczb $[{\alpha \atop \beta}]$. Podobnie jak w innych algorytmach ewolucyjnych, algorytm NPQGA bazuje na zmieniających się w czasie, dynamicznych populacjach rozwiązań i korzysta z funkcji oceniającej te rozwiązania wykorzystując własności bitów kwantowych. Oprócz stosowania tradycyjnie rozumianych operatorów selekcji, krzyżowania oraz mutacji, autorzy zaproponowali również operator katastrofy, a także operator związany z bramkami kwantowymi, służący do zmiany stanów kubitów.

\subsection{Kodowanie rozwiązań}
W publikacji, został zawarty przykład obrazujący działanie algorytmu dla problemu szeregowania zadań. Został również przedstawiony sposób kodowania rozwiązań, który nadaje się także dla problemu przydziału kwadratowego. Ogólnie, rozwiązanie problemu jest ciągiem kubitów i można je przedstawić w następujący sposób:
\newline
\begin{equation}
\left[ \begin{array}{ccc} \alpha_1 \\ \beta_1 \end{array} \right| \left. \begin{array}{ccc} \alpha_2 \\ \beta_2 \end{array}  \right| \left. \begin{array}{ccc} ... \\ ... \end{array}  \right| \left. \begin{array}{ccc} \alpha_l \\ \beta_l \end{array}  \right]
\end{equation}
\newline
gdzie 
\newline
\begin{equation}
l=([\log_2^n] + 1)\cdot n
\end{equation}
\newline
a \textit{n} oznacza rozmiar problemu, czyli z ilu elementów składa się permutacja reprezentująca rozwiązanie problemu. Nawiasy kwadratowe oznaczają cechę liczby.
Niestety, z samego ciągu kubitów nie wynika od razu jaką permutację ten ciąg koduje. By uzyskać rozwiązanie permutacyjne, które jest używane dla problemu QAP, należy wykonać następujące kroki:
\begin{enumerate}
\item dla każdego kubitu wylosuj liczbę $\eta$ z przedziału \textit{[0,1]},
\item jeśli $\eta < |\alpha_i|^2$, to określ stan \textit{i-tego} kubitu na 0, w przeciwnym przypadku na 1,
\item dla utworzonego ciągu bitów, każde $[\log_2^n] + 1$ bitów zamień na postać dziesiętną,
\item mając ciąg liczb naturalnych posortuj go rosnąco z zapamiętaniem pozycji liczb w ciągu,
\item jeśli dwie kolejne liczby są różne, to mniejsza z nich reprezentuje przydzielony do placówki o numerze indeksu obiekt o mniejszym numerze, a jeśli są równe, to liczba z mniejszym indeksem reprezentuje obiekt o niższym numerze. Elementowi o najmniejszej wartości przyporządkuj obiekt o pierwszym numerze.
\item ustaw rosnąco według indeksów powyższy ciąg liczb naturalnych zastępując te liczby odpowiadającymi im numerami przydzielonych obiektów według zasad z punktu piątego.
\end{enumerate}
W ten sposób uzyskana zostaje permutacja, w której pozycja określa numer lokalizacji, a wartość liczby na tej pozycji, określa przydzielony do niej obiekt. Przykładowo, zdekodowany do postaci dziesiętnej ciąg liczb \textit{5, 3, 7, 1, 1, 2, 1} po uszeregowaniu rosnąco wygląda w następujący sposób: \textit{1, 1, 1, 2, 3, 5, 7}, co reprezentuje następujący przydział obiektów: \textit{1, 2, 3, 4, 5, 6, 7}. Po uwzględnieniu, która liczba z ciągu reprezentuje który obiekt, otrzymuje się ostatecznie następującą permutację, reprezentującą rozwiązanie problemu QAP: \textit{6, 5, 7, 1, 2, 4, 3}.

\subsection{Operatory genetyczne}
Jako, że algorytm NPQGA jest modyfikacją algorytmu genetycznego, w swym działaniu korzysta z typowych operatorów genetycznych. Autorzy algorytmu w zaprezentowanym przykładzie zaproponowali selekcję ruletkową, operator krzyżowania CX, mutację polegającą na zamianie w losowym kubicie parametrów $\alpha$ i $\beta$ oraz bramkę kwantową do zmiany stanów kubitów o nazwie \textit{rotation gate}.

Pewnym nietypowym rozwiązaniem związanym z krzyżowaniem jest zmiana prawdopodobieństwa zajścia krzyżowania z czasem. Im więcej iteracji algorytmu minęło, tym mniejsze jest prawdopodobieństwo krzyżowania. Należy ustawić jako parametry algorytmu prawdopodobieństwa maksymalne i minimalne zajścia krzyżowania.
\newline
\begin{equation}
P_c^+= \left\{ \begin{array}{ccc} \frac{P_{c max}}{1+\frac{t}{t_{max}}}, \; P_c^+ > P_c min \\ P_{c max}, \; P_c^+ < P_c min \end{array} \right.
\end{equation}
\newline
Operator CX działa w następujący sposób:
\begin{enumerate}
\item Wybierany jest dowolny element z pierwszego z rodziców, najczęściej jest to pierwszy element permutacji.
\item Sprawdzana jest wartość elementu w drugim rodzicu na pozycji tej samej, co wybrany element w pierwszym rodzicu.
\item Znajdywany jest element w pierwszym rodzicu o wartości sprawdzonej w punkcie 2 i dla tego elementu powtarzamy krok 2.
\item Wykonywane są powyższe kroki, aż do dotarcia w pierwszym rodzicu do punktu startowego.
\item Uzyskane w ten sposób zestawy punktów w obu rodzicach przenoszone są do rozwiązań potomków z zachowaniem indeksów elementów permutacji w taki sposób, że elementy z rodzica pierwszego umieszczane są w potomku nr 2 i na odwrót.
\item Powtarzane jest szukanie punktów poczynając od pierwszego niewybranego punktu w rodzicu pierwszym i znalezione grupy punktów są kopiowane do potomków, lecz tym razem elementy z pierwszego rodzica zostają umieszczone w potomku pierwszym. W następnym wyszukiwaniu ponownie elementy z rodzica pierwszego kopiowane są do potomka drugiego itd.
\item Wyszukiwanie cykli elementów powtarza się aż wszystkie elementy zostaną wybrane.
\end{enumerate}

Mutacja zachodzi wtedy dla danego osobnika, gdy wylosowana dla niego liczba z przedziału \textit{[0,1]} jest mniejsza niż prawdopodobieństwo mutacji $p_m$. Wtedy losuje się, który kubit z rozwiązania poddany będzie modyfikacji, która wygląda w sposób następujący:
\newline
\begin{equation}
\left[ {\alpha_i^\prime \atop \beta_i^\prime} \right] = \left[ {\beta_i \atop \alpha_i} \right]
\end{equation}
\newline
Po dokonaniu powyższej zamiany, należy ponownie sprawdzić stan kubitu, co może się wiązać ze zmianą cyfry dziesiętnej, w której skład wchodzi zmodyfikowany kubit, co dalej może pociągać za sobą zmianę całej permutacji.

Operator bramki kwantowej jest tym elementem algorytmów kwantowych, który ma największy wpływ na zmianę stanów bitów kwantowych. Istnieje wiele różnych odmian bramek kwantowych, takie jak bramka NOT, CNOT, bramka Hadamarda. W algorytmie NPQGA została zaproponowana bramka rotacyjna - \textit{rotation gate}. Uaktualnienie parametrów \textit{$\alpha$} i \textit{$\beta$} następuje w następujący sposób:
\newline
\begin{equation}
\left[ {\alpha_i^\prime \atop \beta_i^\prime} \right] = \begin{bmatrix}
\cos(\Theta_i) & -\sin(\Theta_i) \\ \sin(\Theta_i) & \cos(\Theta_i)
\end{bmatrix}
\end{equation}
\newline
Kąt $\Theta_i$ określony jest poprzez swoją wartość i kierunek obrotu:
\newline
\begin{equation}
\Theta_i=\Delta \Theta_i \cdot s(\alpha_i, \beta_i),
\end{equation}
\newline
gdzie $\Delta \Theta_i$ określa wartość kąta o jaki należy dokonać rotacji, a $s(\alpha_i, \beta_i)$ określa kierunek obrotu. Zarówno wartość kąta jak i jego kierunek  odczytuje się z tablicy \textit{Look Up} i zależą one od najlepszego rozwiązania znalezionego w danej populacji i wartości parametrów $\alpha$ i $\beta$. Sprawdzany jest stan każdego kubitu w rozwiązaniu najlepszym i porównywany ze stanem odpowiadającego mu kubitu w poddawanym działaniu bramki kwantowej rozwiązaniu. Poniżej znajduje się tablica  \textit{Look Up} z wartościami zaproponowanymi przez autorów algorytmu:
\begin{table}[h]
\label{LUT_TAB}
\begin{tabular}{l l l l l l l l}
\hline
$r_i$ & $b_i$ & $f(r)<f(b)$ & $\Delta\Theta_i \cdot \pi$ & $s(\alpha_i,\beta_i)$ & & & \\
\cline{5-8} 
& & & & $\alpha_i \cdot \beta_i > 0$ & $\alpha_i \cdot \beta_i < 0$ & $\alpha_i = 0$ & $\beta_i = 0$ \\
\hline
0 & 0 & False & $0.2\pi$ & 0 & 0 & 0 & 0\\
0 & 0 & True  & 0        & 0 & 0 & 0 & 0\\
0 & 1 & False & $0.5\pi$ & 0 & 0 & 0 & 0\\
0 & 1 & True  & 0        & -1 & +1 & +1 lub -1 & 0\\
1 & 0 & False & $0.5\pi$ & -1 & +1 & +1 lub -1 & 0\\
1 & 0 & True  & 0        & +1 & -1 & 0 & +1 lub -1\\
1 & 1 & False & $0.2\pi$ & +1 & -1 & 0 & +1 lub -1\\
1 & 1 & True  & 0        & +1 & -1 & 0 & +1 lub -1\\
\hline
\end{tabular}
\caption{LUT dla bramki kwantowej}
\end{table}

W przypadku gdy stany porównywanych kubitów są różne i wybrane rozwiązanie jest gorsze niż najlepsze dotychczas, proponowana jest zmiana o większy kąt, a gdy mają taką samą wartość, to zaleca się kąt o mniejszej wartości. Kierunek obrotu zależy od iloczynu prawdopodobieństw, że kubit znajduje się w stanie \textit{0} i \textit{1}. W przypadku problemu szeregowania zadań z minimalizacją czasu wykonań wszystkich z nich, rozwiązanie lepsze ma mniejszą wartość funkcji celu, dlatego kąt zmieniany jest, gdy w trzeciej kolumnie tabeli znajduje się wartość \textit{False}. Celem działania operatora \textit{rotation gate} jest  modyfikacja rozwiązań w danym pokoleniu, dzięki której będą one bardziej podobne do rozwiązania najlepszego. Rozwiązanie poddane działaniu tego operatora z większym prawdopodobieństwem będzie podobne do osobnika najlepszego.

Autorzy algorytmu zaproponowali również tak zwany operator katastrofy. Jest on wykorzystywany w sytuacji, w której nie uzyskiwana jest poprawa rozwiązania podczas określonej liczby iteracji algorytmu i skutkuje ponownym zainicjowaniem populacji. Zakłada się, że zdarzenie to spowodowane jest znalezieniem ekstremum lokalnego.

\subsection{Równoległość algorytmu}
Ciekawym elementem algorytmu jest sposób w jaki dokonywany jest przegląd rozwiązań. Zaproponowany został model, w którym istnieje wiele równoległych populacji pogrupowanych w tak zwane uniwersa. W jednym uniwersum znajduje się wiele populacji. Wymiana informacji obydwa się na dwóch poziomach:
\begin{enumerate}
\item pomiędzy uniwersami,
\item pomiędzy populacjami w danym uniwersum.
\end{enumerate}

Wymiana informacji w obrębie jednego uniwersum wzorowana jest na osmozie. Informacje o dobrych rozwiązaniach wędrują w jedną stronę,w kierunku populacji, dla której suma dopasowania rozwiązań jest gorsza. Natomiast wymiana informacji pomiędzy uniwersami bazuje na czasowej zmianie wartości, do której dążą rozwiązania w innych uniwersach. Po tej wymianie, rozwiązania z jednego uniwersum jako cel swego rozwoju obierają optymalny kierunek innego i na odwrót.
Obie powyższe strategie zachodzą z ustalaną częstotliwością. Autorzy podają \textit{10\%-20\%} wszystkich iteracji jako typową wartość tego parametru. 

\section{Pseudokod algorytmu NPQGA}
Poniżej znajduje się pseudokod algorytmu:
\newpage
\begin{framed}
\begin{algorithm}[H]
	Wczytaj parametry algorytmu\;
	Zainicjuj populacje, wyznacz ich permutacyjna postać, policz ich dopasowanie, zapisz najlepszy rezultat\;
 	\While{nie wystąpił warunek stopu}
 	{
 		$t\leftarrow t+1$\;
  		\For{dla każdej populacji}
  		{
  			Wybierz najlepsze rozwiązanie z populacji\;
  			Dokonaj krzyżowania i mutacji\;
  			\If{zaistniały warunki dla operatora katastrofy}
  			{
  				Użyj operatora katastrofy do wygenerowania następnego pokolenia\;
  			}
  			\Else
  			{
  				Użyj bramki kwantowej do wygenerowania następnego pokolenia\;
  			}
  		}
  		\For{dla każdej populacji w każdy uniwersum}
  		{
  			Dokonaj wymiany informacji między populacjami\;
  		}
  		\For{dla każdego uniwersum}
  		{
  			Dokonaj wymiany informacji między uniwersami;
  		}
 	}
 	\caption{Algorytm NPQGA}
\end{algorithm}
\end{framed}
Powyższy schemat działania odnosi się do wersji algorytmu wykorzystanej przez jego autorów do rozwiązania problemu szeregowania zadań. Jednym z celów niniejszej pracy było stworzenie aplikacji wykorzystującej wybrany algorytm przybliżony do rozwiązywania problemu przydziału kwadratowego. Powyższy algorytm został zaimplementowany, lecz z pewnymi modyfikacjami. Zostały one przedstawione w następnym rozdziale.