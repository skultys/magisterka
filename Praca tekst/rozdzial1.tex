\chapter{Wstęp}
\label{cha:wstep}
Nadejście rewolucji przemysłowej spowodowało powstanie wielkiej liczby firm, przedsiębiorstw, które były dużo większe niż znane wcześniej zakłady rzemieślnicze. Ich rozmiar powodował również rozrost złożoności problemów związanych z organizacją tychże firm. W związku z kompleksowością pojawił się problem jak najlepszego przydziału dostępnych zasobów, najwłaściwszej organizacji pracy. Zaistniała więc potrzeba stworzenia różnych metod, sposobów, dzięki którym można by powyższe problemy w jakiś sposób rozwiązać. Ta potrzeba doprowadziła do powstania badań operacyjnych.

Chociaż początki badań operacyjnych faktycznie związane są z rewolucją przemysłową, to jednak pojęcie badań operacyjnych, które znamy obecnie, związane jest z działaniami podejmowanymi przez agencje wojskowe już na początku drugiej wojny światowej. Można by stwierdzić, że pewną ironią losu jest fakt, iż wiele wykorzystywanych dzisiaj odkryć i wynalazków, które bardzo ułatwiają nam  codziennie życie, zostało powołanych do życia w związku z działaniami, które najczęściej kojarzą się z cierpieniem i przemocą. 
Brytyjskie i amerykańskie organizacje wojskowe zatrudniły ogromną liczbę naukowców, by ci wdrożyli naukowe podejście do spraw związanych z efektywnym zbrojeniem się, zarządzaniem zasobami oraz taktycznymi i strategicznymi problemami związanymi z prowadzeniem działań wojennych.

Mówi się, że podjęte wysiłki miały duży wpływ na takie znane wydarzenia jak Bitwa o Anglię, czy też Bitwa o Atlantyk.

Sukces, jaki odniosły badania operacyjne w wojskowości, zachęcił ludzi związanych z przemysłem do zaadaptowania ich również w samym przemyśle. Ożywienie w gospodarce, spowodowane zakończeniem wojny, doprowadziło do wzrostu złożoności działalności firm, a więc badania operacyjne idealnie nadawały się jako narzędzie wspierające organizację i zarządzanie tymi przedsiębiorstwami. 

Niewątpliwie, następujący szybki rozwój badań operacyjnych miał swą przyczynę  w tym, że wielu naukowców, którzy parali się nimi podczas wojny, szukając pracy w swojej branży, chętnie zajęło się dalszymi studiami nad badaniami operacyjnymi w dziedzinach związanych nie tylko z wojskowością. Oczywiście nie oznaczało to, że wojsko całkowicie zrezygnowało z badań operacyjnych. Również postęp związany z powstaniem komputerów dał odpowiednie narzędzia do analizy coraz bardziej złożonych problemów. Wiele problemów związanych z podejmowaniem decyzji, wyborem najlepszego rozwiązania można było rozwiązać podpierając się matematycznym modelem. Mając więc problem w sformalizowanej postaci można zaproponować algorytm, który rozwiąże dane zagadnienie. Sam algorytm jako ciąg kolejnych instrukcji, które należy wykonać, by osiągnąć dany cel, bardzo dobrze nadaje się do zaimplementowania i wykonania na komputerze. Coraz szybsze komputery o coraz pojemniejszych pamięciach, a także wykorzystanie technik programowania równoległego i współbieżnego pozwalają na rozwiązywanie coraz bardziej złożonych problemów w rozsądnym czasie. Z czasem więc zaczęły się pojawiać kolejne algorytmy, ale też nowe problemy. Również dokonywane odkrycia naukowe pozwoliły na wykorzystanie występujących w naturze procesów do tworzenia nowatorskich metod rozwiązywania skomplikowanych zagadnień.

Niestety, istnieje wiele problemów, w przypadku których można jedynie powiedzieć, że mają optymalne rozwiązanie, nie da się jednak znaleźć go przy wykorzystaniu obecnie dostępnej technologii. Poprzez oszacowanie złożoności obliczeniowej algorytmów można tylko stwierdzić, że potrzebny czas do znalezienia rozwiązania problemu przy ich wykorzystaniu jest niejednokrotnie dłuższy niż przeciętny czas życia człowieka. Przykładem takiego zagadnienia jest tzw. problem przydziału kwadratowego, polegającego na przydziale pewnej liczby placówek do takiej samej liczby miejsc. Wynika z tego, że dla n placówek możliwe jest w sumie n! wszystkich permutacji. Wraz ze wzrostem liczby placówek, które należy przydzielić, ilość możliwych rozwiązań rośnie bardzo szybko. Już dla stosunkowo małej ilości placówek możliwa jest ogromna liczba rozwiązań. Istnieje  więc wiele algorytmów przybliżonych, inaczej nazywanych aproksymujących, które znajdują jedynie przybliżone rozwiązanie postawionego problemu. Nie oznacza to jednak, że zwrócone przez algorytm rozwiązanie nie może być faktycznie optymalne, jednak nie da się przeważnie tego sprawdzić.

Jak już zostało to nadmienione wyżej, istnieje wiele algorytmów wykorzystujących analogie do zachowań występujących w przyrodzie. Przykładem są algorytmy genetyczne, których działanie wzorowane jest na ewolucji biologicznej - spośród znalezionych w danym pokoleniu rozwiązań, wybierane są najlepsze z nich (według pewnych ustalonych dla danego problemu kryteriów), traktowane są jako rodzice dla następnego pokolenia, które dziedziczy po rodzicach ich cechy. Wykorzystywane są również różnego rodzaju operatory mutacji, katastrofy itp. 

\section{Cel pracy}
\label{sec:cel}
Celem niniejszej pracy jest właśnie omówienie wybranych algorytmów  przybliżonych, ich wad i zalet oraz przedstawienie ich wykorzystania w kontekście problemu przydziału kwadratowego. Na potrzeby pracy napisana została również aplikacja komputerowa wykorzystująca jeden z zaprezentowanych algorytmów aproksymujących i rozwiązująca problem przydziału kwadratowego.

\section{Zawartość pracy}
\label{sec:zawartosc}
W kolejnych rozdziałach znajduje się dokładny opis problemu przydziału kwadratowego, przedstawienie dziedzin, w których tenże problem można wykorzystać. Następnie znajduje się przegląd algorytmów przybliżonych: sposób ich działania, wady i zalety, opis tego w jaki sposób można by je wykorzystać, by rozwiązać wyżej wspomniany problem. Dalej znajduje się opis napisanej aplikacji i procesu samej implementacji oraz wyjaśnienie wyboru danego algorytmu. Kolejne rozdziały zawierają przedstawienie wykonanych testów i ich rezultaty oraz dokonaną statystykę na ich podstawie.
