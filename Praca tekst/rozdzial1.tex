\chapter{Wstęp}
\label{cha:wstep}
Nadejście rewolucji przemysłowej spowodowało powstanie wielkiej liczby firm, przedsiębiorstw, które były dużo większe niż znane wcześniej zakłady rzemieślnicze. Ich rozmiar powodował również rozrost złożoności problemów związanych z organizacją tychże firm. W związku z kompleksowością pojawiły się problemy jak najlepszego przydziału dostępnych zasobów oraz najwłaściwszej organizacji pracy. Zaistniała więc potrzeba stworzenia różnych metod, dzięki którym można by powyższe problemy w jakiś sposób rozwiązać. Ta potrzeba doprowadziła do powstania badań operacyjnych.

Chociaż początki badań operacyjnych faktycznie związane są z rewolucją przemysłową, to jednak pojęcie badań operacyjnych, które znamy obecnie, związane jest z działaniami podejmowanymi przez agencje wojskowe już na początku drugiej wojny światowej. Można by stwierdzić, że pewną ironią losu jest fakt, iż wiele wykorzystywanych dzisiaj odkryć i wynalazków, które bardzo ułatwiają nam  codziennie życie, zostało powołanych do życia w związku z działaniami kojarzącymi się najczęściej z cierpieniem i przemocą. 
Brytyjskie i amerykańskie organizacje wojskowe zatrudniły ogromną liczbę naukowców, by ci wdrożyli naukowe podejście do spraw związanych z efektywnym zbrojeniem się, zarządzaniem zasobami oraz taktycznymi i strategicznymi problemami związanymi z prowadzeniem działań wojennych.

Mówi się, że podjęte wysiłki miały duży wpływ na takie znane wydarzenia jak Bitwa o Anglię, czy też Bitwa o Atlantyk.

Sukces, jaki odniosły badania operacyjne w wojskowości, zachęcił ludzi związanych z przemysłem do zaadaptowania ich również w samym przemyśle. Ożywienie w gospodarce, spowodowane zakończeniem wojny, doprowadziło do wzrostu złożoności działalności firm, a więc badania operacyjne idealnie nadawały się jako narzędzie wspierające organizację i zarządzanie tymi przedsiębiorstwami. 

Niewątpliwie następujący szybki rozwój badań operacyjnych miał swą przyczynę  w tym, że wielu naukowców, którzy parali się nimi podczas wojny, szukając pracy w swojej branży, chętnie zajęło się dalszymi studiami nad badaniami operacyjnymi w dziedzinach związanych nie tylko z wojskowością. Oczywiście nie oznaczało to, że wojsko całkowicie zrezygnowało z badań operacyjnych. Również postęp związany z powstaniem komputerów dał odpowiednie narzędzia do analizy coraz bardziej złożonych problemów. Wiele problemów związanych z podejmowaniem decyzji, wyborem najlepszego rozwiązania można było rozwiązać podpierając się matematycznym modelem. Mając więc problem w sformalizowanej postaci można zaproponować algorytm, który rozwiąże dane zagadnienie. Sam algorytm jako ciąg kolejnych instrukcji, które należy wykonać, by osiągnąć dany cel, bardzo dobrze nadaje się do zaimplementowania i wykonania na komputerze. Coraz szybsze komputery o coraz pojemniejszych pamięciach, a także wykorzystanie technik programowania równoległego i współbieżnego pozwalają na rozwiązywanie coraz bardziej złożonych problemów w rozsądnym czasie. Z czasem więc zaczęły się pojawiać kolejne algorytmy, ale też nowe problemy. Również dokonywane odkrycia naukowe pozwoliły na wykorzystanie występujących w naturze procesów do tworzenia nowatorskich metod rozwiązywania skomplikowanych zagadnień.

Niestety, istnieje wiele problemów, w przypadku których można jedynie powiedzieć, że mają optymalne rozwiązanie, nie da się jednak znaleźć go przy wykorzystaniu obecnie dostępnej technologii. Poprzez oszacowanie złożoności obliczeniowej algorytmów można tylko stwierdzić, że potrzebny czas do znalezienia rozwiązania problemu przy ich wykorzystaniu jest niejednokrotnie dłuższy niż przeciętny czas życia człowieka. Przykładem takiego zagadnienia jest tzw. problem przydziału kwadratowego, polegającego na przydziale pewnej liczby placówek do takiej samej liczby miejsc. Wynika z tego, że dla n placówek możliwe jest w sumie \textit{n!} wszystkich permutacji. Wraz ze wzrostem liczby placówek, które należy przydzielić, liczba możliwych rozwiązań rośnie bardzo szybko. Już dla stosunkowo małej liczby placówek możliwa jest ogromna liczba rozwiązań. Istnieje  więc wiele algorytmów przybliżonych, inaczej nazywanych aproksymacyjnych, które znajdują jedynie przybliżone rozwiązanie postawionego problemu. Nie oznacza to jednak, że zwrócone przez algorytm rozwiązanie nie może być faktycznie optymalne, ale przeważnie nie da się tego sprawdzić.

Jak już zostało to nadmienione wyżej, istnieje wiele algorytmów wykorzystujących analogie do zachowań występujących w przyrodzie. Przykładem są algorytmy genetyczne, których działanie wzorowane jest na ewolucji biologicznej - spośród znalezionych w danym pokoleniu rozwiązań, wybierane są najlepsze z nich (według pewnych ustalonych dla danego problemu kryteriów), traktowane są jako rodzice dla następnego pokolenia, które dziedziczy po rodzicach ich cechy. Wykorzystywane są również różnego rodzaju operatory mutacji, katastrofy itp. 

\section{Cel pracy}
\label{sec:cel}
Celem niniejszej pracy jest dokonanie przeglądu wybranych algorytmów przybliżonych, ich wad i zalet oraz przedstawienie ich wykorzystania w kontekście problemu przydziału kwadratowego. Następnie, przy użyciu specjalnie napisanej na potrzeby pracy aplikacji, która rozwiązuje problem przydziału kwadratowego, należy zaprezentować rezultaty przeprowadzonych eksperymentów oraz opisać zastosowane scenariusze testowe i dokonać analizy otrzymanych wyników.

\section{Zawartość pracy}
\label{sec:zawartosc}
Rozdział nr 2 zawiera opis problemu przydziału kwadratowego, obszar jego zastosowań i jego model matematyczny.

W rozdziale trzecim zostały przedstawione wybrane algorytmy aproksymacyjne, geneza ich powstania oraz wybrane sposoby ich wykorzystania w celu rozwiązania problemu przydziału kwadratowego.

Rozdział czwarty poświęcony jest idei kwantowych algorytmów ewolucyjnych, a także opisano w nim wybrany algorytm kwantowy NPQGA, który został zaimplementowany w aplikacji będącej jednym z celów niniejszej pracy. Wprowadzone zmiany i modyfikacje we wspomnianym algorytmie są tematem kolejnego, piątego rozdziału pracy.

Następny, szósty rozdział zawiera opis utworzonej aplikacji, informacje o działaniu algorytmu i obsłudze programu.

Rozdział siódmy omawia metodykę eksperymentów przeprowadzanych przy wykorzystaniu napisanej aplikacji. Zawarte w nim są opisy wybranych instancji testowych, a także przedstawione są scenariusze testowe.

Rozdział ósmy  poświęcony jest faktycznie przeprowadzonym eksperymentom, zawiera informacje o tym, jakie ustawienia parametrów algorytmu były testowane i porównywane dla wybranych instancji testowych oraz przedstawione są w nim rezultaty eksperymentów w postaci różnych tabel i wykresów.

Rozdział dziewiąty skupia się na analizie rezultatów uzyskanych z przeprowadzonych testów. Opisane są wnioski dotyczące poszczególnych eksperymentów.

Ostatni, dziesiąty rozdział poświęcony jest ogólnym wnioskom dotyczącym tematyki całej pracy oraz jej podsumowaniu.