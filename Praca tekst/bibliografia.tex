\bibliographystyle{alpha}
\bibliography{bibliografia}
\begin{thebibliography}{99}

\bibitem{ANT_DISC} Dorigo, Marco, Di Caro, Gambardella, Gianni, Luca M. \textit{Ant Algorithms for Discrete Optimization}.

\bibitem{FIL_ROJOWE} Filipowicz Bogusław, Chmiel Wojciech, Kadłuczka Piotr, \textit{Ukierunkowane poszukiwanie przestrzeni rozwiązań w algorytmach rojowych}, Półrocznik Automatyka, z. 2m t. 13, Wydawnictwo AGH, Kraków 2009.

\bibitem{FILIP_STADNE} Filipowicz Bogusław, Kwiecień Joanna, \textit{Algorytmy stadne w optymalizacji problemów przydziału przy kwadratowym wskaźniku jakości (QAP)}, Półrocznik Automatyka, z. 2, t. 15, Wydawnictwo AGH, Kraków 2011.

\bibitem{TABU} Glover Fred, \textit{Tabu Search: A Tutorial}.

\bibitem{TABU_P1} Glover Fred, \textit{Tabu Searcg - Part I}, ORSA Journal on Computing, nr 3, t. 1, 1989.

\bibitem{GOLDBERG} Goldberg David E., \textit{Algorytmy genetyczne i ich zastosowania}, przeł. Kazimierz Grygiel, Wyd 3, Wydawnictwa Naukowo - Techniczne, Warszawa 2003.

\bibitem{NPQGA} Gu Jinwei, Gu Xingsheng, Gu Manzhan, \textit{A novel parallel quantum genetic algorithm for stochastic job shop scheduling}, Journal of Mathematical Analysis and Applications, 63-81, 2009.

\bibitem{AE_GWIAZDA} Gwiazda Tomasz Dominik, \textit{ ewolucyjne w rozwiązywaniu nieliniowych problemów decyzyjnych}, Wydawnictwa Naukowe Wydziału Zarządzania Uniwersytetu Warszawskiego, Warszawa 2002.

\bibitem{INTRO} Hiller Frederick S., Lieberman Gerald J., \textit{Introduction to Operations Research}, Wyd 5, McGraw - Hill Publishing Company, Nowy Jork 1990.

\bibitem{APROX_JONES} Jones Gareth, \textit{Genetic and Evolutionary Algorithms}.

\bibitem{QAP_DEF_CHMIEL} Kadłuczka Piotr, Chmiel Wojciech, \textit{Efektywność algorytmu ewolucyjnego wykorzystującego warunkową wartość oczekiwaną funkcji celu}, Półrocznik Automatyka, z. 1-2, t. 9, Wydawnictwo AGH, Kraków 2005.

\bibitem{PSO} Kennedy James, Eberhart Russel, \textit{Particle Swarm Optimization}, 1995.

\bibitem{SWARM_INT} Kennedy James, Eberhart Russel C., \textit{Swarm Intelligence}, Morgan Kaufmann Publishers, San Francisco, 2001.

\bibitem{COMPARISON} Laboudi Zakaria, Chikhi Salim,  \textit{Comparison of Genetic Algorithm and Quantum
Genetic Algorithm}, The International Arab Journal of Information Technology, nr 3, t. 9, 2012.

\bibitem{PSO2} Liu Hongbo, Abraham Ajith, Zhang Jianying \textit{A Particle Swarm Approach to Quadratic Assignment Problems}.

\bibitem{MICH_AG} Michalewicz Zbigniew, \textit{Algorytmy genetyczne + struktury danych = programy ewolucyjne}, przeł. Zbigniew Nahorski, Wyd 3, Wydawnictwa Naukowo-  Techniczne, Warszawa 2003.

\bibitem{MICH_JAK} Michalewicz Zbigniew, Fogel David B., \textit{jak to rozwiązać czyli nowoczesna metaheurystyka}, przeł. Aleksy Schubert, Joanna Schubert, Wydawnictwa Naukowo - Techniczne, Warszawa 2006.

\bibitem{META_HEU} Stawowy Adam, \textit{Heurystyki i metaheurystyki}, wykład z przedmiotu Inteligencja Obliczeniowa.

\bibitem{ACO_DORIGO} St\"utzle Thomas, Dorigo Marco, \textit{ACO Algorithms for the Quadratic Assignment Problem}.

\bibitem{APROX} Vizirani Vijay V., \textit{Approximation Algorithms}, Springer, Berlin 2003.

\bibitem{INSTANCJE} Zbiór instancji testowych problemu QAP: http://anjos.mgi.polymtl.ca/qaplib/inst.html.

\end{thebibliography}