\chapter{Zagadnienie przydziału kwadratowego}
\label{cha:qap}

\section{Opis problemu}
\label{sec:opis}
Zagadnienie przydział kwadratowego (Qadratic Assignment Problem - QAP) jest jednym najtrudniejszych problemów optymalizacji kombinatorycznej. Należy on klasy problemów NP - trudnych i dla rozmiarów o wartości większej niż 30 wymagane jest stosowanie algorytmów przybliżonych w celu jego rozwiązania. Zagadnienia przydziału kwadratowego zostało przedstawione przez Koopmansa i Beckmanna w roku 1957. Problem ten jest matematycznym modelem sytuacji, w której chcemy przydzielić pewną ilość placówek do takiej samej ilości lokalizacji (miejsc) znając przy tym odległości pomiędzy danymi lokalizacjami oraz przepływu miedzy placówkami. Przydziału tego należy dokonać minimalizując koszt tej operacji, który jest proporcjonalny do przepływu pomiędzy placówkami pomnożonego przez odległość między miejscami, do których te placówki zostały przydzielone. Z racji, iż trudność rozwiązania tego problemu jest duża oraz, że modeluje on wiele faktycznych zagadnień, wielu autorów poświęciło mu dużo uwagi, przez co znaleźć można wiele różnych publikacji traktujących o problemie QAP.

\section{Obszary zastosowań}
\label{sec:zastosowanie}
Przy pomocy problemu przydziału kwadratowego można modelować wiele różnych zagadnień, które występują wokoło. Do dziedzin,w których zagadnienie QAP znajduje zastosowanie należą m. in:
\begin{itemize}
\item ekonomia,
\item informatyka,
\item elektronika, np. projektowanie układów elektroniki,
\item logistyka, np. lokalizacja współpracujących ze sobą fabryk, zakładów produkcyjnych, oddziałów w szpitalach,
\item mechanika - wyważanie turbin w silnikach odrzutowych.
\end{itemize}